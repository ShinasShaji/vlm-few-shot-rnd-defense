\documentclass[lettersize,journal]{IEEEtran}
\usepackage[table]{xcolor}
\usepackage{amsmath,amssymb,amsfonts}
\usepackage{algorithm}
\usepackage{array}
\usepackage{subcaption}
\usepackage{textcomp}
\usepackage{stfloats}
\usepackage{url}
\usepackage{verbatim}
\usepackage{graphicx}
\usepackage{cite}
\usepackage{subfiles}
\usepackage[hidelinks]{hyperref}
\usepackage{amsfonts}
\usepackage{booktabs}
\usepackage{multirow}
\usepackage{pifont}
\usepackage{float}
\usepackage{algpseudocode}
\usepackage{siunitx}
\usepackage[framemethod=TikZ]{mdframed}
\usepackage[bb=dsserif]{mathalpha}
\usepackage{tabularx}
\usepackage{enumitem}
\usepackage[acronym,automake]{glossaries-extra}

% Custom styles for prompts and code blocks
\usepackage[most]{tcolorbox}

% Define a small monospace text command
\newcommand{\smalltt}[1]{\small{\texttt{#1}}\normalsize}

% Define colors for different prompt types
\definecolor{promptbg}{RGB}{250, 250, 250}
\definecolor{promptframe}{RGB}{200, 200, 200}
\definecolor{codebg}{RGB}{245, 245, 245}
\definecolor{placeholder}{RGB}{100, 100, 100}

% Command for placeholder text
\newcommand{\placeholder}[1]{\textcolor{placeholder}{\texttt{\{#1\}}}}

% Define a nice prompt box style
\newtcolorbox{promptbox}[1][]{
    enhanced,
    colback=promptbg,
    colframe=promptframe,
    boxrule=0.5pt,
    arc=3pt,
    outer arc=3pt,
    leftrule=2pt,
    fontupper=\footnotesize\ttfamily,
    left=5pt,
    right=5pt,
    top=5pt,
    bottom=5pt,
    #1
}

\setabbreviationstyle[acronym]{long-short}
\subfile{subfiles/acronyms}
\makeglossaries{}


\definecolor{hbrsblue}{RGB}{1, 106, 186}
\definecolor{beigefocus}{RGB}{240, 237, 207}
\definecolor{darkgreen}{RGB}{20, 100, 20}
\definecolor{darkred}{RGB}{120, 30, 30}

\mdfdefinestyle{frameStyle}{%
	roundcorner=5pt,
	backgroundcolor=white}

\renewcommand{\arraystretch}{1.5}
\renewcommand{\vec}[1]{\mathbf{#1}}
\newcommand{\cmark}{\ding{51}}%
\newcommand{\xmark}{\ding{55}}%
\newcolumntype{P}[1]{>{\centering\arraybackslash}p{#1}}
\newcolumntype{M}[1]{>{\centering\arraybackslash}m{#1}}
\newcolumntype{L}[1]{>{\raggedright\arraybackslash\hspace{0pt}}m{#1}}

\newtheorem{definition}{Definition}

\algtext*{EndIf}
\algtext*{EndFor}
\algtext*{EndWhile}
\algtext*{EndFunction}

\DeclareMathOperator*{\argmax}{arg\,max}

% updated with editorial comments 8/9/2021

\begin{document}

\title{Evaluation of Few-Shot Transfer of Vision-Language Foundation Models to Learn Lightweight Models for Robotic Vision Tasks}

\author{Shinas Shaji
        % <-this % stops a space
\thanks{$^*$Submitted to the Department of Computer Science at Hochschule Bonn-Rhein-Sieg in partial fulfilment of the requirements for the degree of Master of Science in Autonomous Systems}
\thanks{$^{\dagger}$Supervised by Prof. Dr. Sebastian Houben (Hochschule Bonn-Rhein-Sieg, Fraunhofer IAIS) and Santosh Thoduka M.Sc. (Fraunhofer IAIS)}
\thanks{$^{\ddagger}$Submitted in February 2025}} %
% \thanks{Manuscript received October Day, 2022; revised Month Day, 2023.}}

\markboth{R\&D Project Report, Master of Autonomous Systems, Hochschule Bonn-Rhein-Sieg}%
{Shinas Shaji: Evaluation of Few-Shot Transfer of Vision-Language Foundation Models to Learn Lightweight Models for Robotic Vision Tasks}

\maketitle

\subfile{subfiles/abstract}

\begin{IEEEkeywords}
% Write up to three keywords about your work.
Few-Shot Transfer, In-context Learning, Prompt Engineering, Vision-Language Models, Image Classification
\end{IEEEkeywords}

% Reset all acronyms to ensure they are defined in the abstract and in the rest of the document
\glsresetall

% Add your content by filling out these
\subfile{subfiles/introduction}
\subfile{subfiles/related_work}
% \subfile{subfiles/background}
\subfile{subfiles/methodology}
\subfile{subfiles/evaluation}
\subfile{subfiles/conclusions}

\bibliographystyle{IEEEtran}
\bibliography{rnd}

\subfile{subfiles/acknowledgement}
\subfile{subfiles/statement_of_originality}
\printglossaries
\subfile{subfiles/appendix}

\end{document}
